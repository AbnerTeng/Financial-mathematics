\documentclass[12pt]{article}
\usepackage{tikz}
\usetikzlibrary{trees}
\usepackage[margin=1in]{geometry} 
\usepackage{amsmath,amsthm,amssymb,scrextend}
\usepackage{fancyhdr}
\setlength{\headheight}{14.5pt}
\addtolength{\topmargin}{-2.5pt}
\pagestyle{fancy}

\newcommand{\cont}{\subseteq}
\usepackage{tikz}
\usepackage{pgfplots}
\usepackage{amsmath}
\usepackage[mathscr]{euscript}
\let\euscr\mathscr \let\mathscr\relax% just so we can load this and rsfs
\usepackage[scr]{rsfso}
\usepackage{amsthm}
\usepackage{amssymb}
\usepackage{multicol}
\usepackage[colorlinks=true, pdfstartview=FitV, linkcolor=blue,
citecolor=blue, urlcolor=blue]{hyperref}

\DeclareMathOperator{\arcsec}{arcsec}
\DeclareMathOperator{\arccot}{arccot}
\DeclareMathOperator{\arccsc}{arccsc}
\newcommand{\ddx}{\frac{d}{dx}}
\newcommand{\dfdx}{\frac{df}{dx}}
\newcommand{\ddxp}[1]{\frac{d}{dx}\left( #1 \right)}
\newcommand{\dydx}{\frac{dy}{dx}}
\let\ds\displaystyle
\newcommand{\intx}[1]{\int #1 \, dx}
\newcommand{\intt}[1]{\int #1 \, dt}
\newcommand{\defint}[3]{\int_{#1}^{#2} #3 \, dx}
\newcommand{\imp}{\Rightarrow}
\newcommand{\un}{\cup}
\newcommand{\inter}{\cap}
\newcommand{\ps}{\mathscr{P}}
\newcommand{\set}[1]{\left\{ #1 \right\}}
\newtheorem*{sol}{Solution}
\newtheorem*{claim}{Claim}
\newtheorem{problem}{Problem}
\pgfplotsset{compat=1.17}
\begin{document}
 
% Don't change the above session

\lhead{Financial Mathematics hw1}
\chead{111352027}
\rhead{\today}
 
% \maketitle
\section{$\sigma$-field}
\tikzstyle{level 1}=[level distance=3.5cm, sibling distance=3.5cm]
\tikzstyle{level 2}=[level distance=3.5cm, sibling distance=2cm]

% Define styles for bags and leafs
\tikzstyle{bag} = [text width=4em, text centered]
\tikzstyle{end} = [circle, minimum width=3pt,fill, inner sep=0pt]

% The sloped option gives rotated edge labels. Personally
% I find sloped labels a bit difficult to read. Remove the sloped options
% to get horizontal labels. 
\begin{tikzpicture}[grow=right, sloped]
\node[bag] {$\Omega$}
    child {
        node[bag] {$\{\omega_4, \omega_5, \omega_6\}$}        
            child {
                node[bag] {$\{\omega_5, \omega_6\}$}
                    child{
                        node[end, label=right:
                            {$\{\omega_6\}$}] {}
                        edge from parent
                    }
                    child {
                        node[end, label=right:
                            {$\{\omega_5\}$}] {}
                        edge from parent
                    }
            }
            child {
                node[end, label=right:
                    {$\{\omega_4\}$}] {}
                edge from parent
            }
            edge from parent 
    }
    child {
        node[bag] {$\{\omega_1, \omega_2, \omega_3\}$}        
        child {
                node[bag] {$\{\omega_2, \omega_3\}$}
                child{
                    node[end, label=right:
                        {$\{\omega_2\}$}] {}
                    edge from parent
                }
                child {
                    node[end, label=right:
                        {$\{\omega_3\}$}] {}
                    edge from parent
                }
            }
            child {
                node[end, label=right:
                    {$\{\omega_1\}$}] {}
                edge from parent
            }
        edge from parent         
    };
\end{tikzpicture}
\newline
\begin{align*}
    \mathcal{F}_0, \mathcal{F}_1 &= \sigma(X_0, X_1) = \{\{\omega_1, \omega_2, \omega_3\}, \{\omega_4, \omega_5, \omega_6\}, \phi, \Omega\}\\
    \mathcal{F}_2 = \sigma(X_0, X_1, X_2) &= \{\omega_1, \omega_4, \{\omega_2, \omega_3, \omega_4, \omega_5, \omega_6\}, \{\omega_1, \omega_2, \omega_3, \omega_5, \omega_6\}\{\omega_1, \omega_2, \omega_3\}, \{\omega_4, \omega_5, \omega_6\},\\
    &\ \ \ \ \{\omega_2, \omega_3, \omega_5, \omega_6\}, \{\omega_1, \omega_4\}, \{\omega_1, \omega_4, \omega_5, \omega_6\}, \{\omega_2, \omega_3\}, \{\omega_5, \omega_6\}, \{\omega_1, \omega_2, \omega_3, \omega_4\},\\
    &\ \ \ \ \{\omega_1, \omega_5, \omega_6\}, \{\omega_2, \omega_3, \omega_4\}\phi, \Omega\}\\
    \mathcal{F}_3 = \sigma(X_0, X_1, X_2, X_3) &= \{\omega_1, \omega_2,\cdots, \phi, \Omega\}
\end{align*}

\section{Stopping time}
\begin{align*}
    \tau_1(\omega_i)=\begin{cases}
        1, & i = 1, 2, 3\\
        2, & i = 4\\
        3, & i = 5, 6
    \end{cases}\quad\ 
    \tau_2(\omega_i)=\begin{cases}
        1, & i = 1\\
        2, & i = 2, 3\\
        3, & i = 4, 5, 6
    \end{cases}
\end{align*}
$[\tau \leq t] = \{\omega \in \Omega|\tau(\omega)\leq t\}\in \mathcal{F}_t$\\
$[\tau_1\leq1] = \{\omega \in \Omega|\tau_1(\omega)\leq 1\} = \{\omega_1, \omega_2, \omega_3\}\in\mathcal{F}_1$\\
$[\tau_1\leq2] = \{\omega \in \Omega|\tau_1(\omega)\leq 2\} = \{\omega_1, \omega_2, \omega_3, \omega_4\}\in\mathcal{F}_2$\\
$[\tau_1\leq3] = \{\omega \in \Omega|\tau_1(\omega)\leq 3\} = \Omega \in\mathcal{F}_3$
\newline
\newline
\quad\ \quad\ $[\tau_2\leq1] = \{\omega \in \Omega|\tau_2(\omega)\leq 1\} = \{\omega_1\}\notin\mathcal{F}_1$\\
\subsection*{Solution}
$\tau_1$ is stopping time and $\tau_2$ isn't
\end{document}