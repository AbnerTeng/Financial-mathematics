\documentclass[12pt]{article}
\usepackage{tikz}
\usetikzlibrary{trees}
\usepackage[margin=1in]{geometry} 
\usepackage{amsmath,amsthm,amssymb,scrextend}
\usepackage{fancyhdr}
\setlength{\headheight}{14.5pt}
\addtolength{\topmargin}{-2.5pt}
\pagestyle{fancy}

\newcommand{\cont}{\subseteq}
\usepackage{graphicx}
\usepackage{float}
\usepackage{tikz}
\usepackage{pgfplots}
\usepackage{amsmath}
\usepackage[mathscr]{euscript}
\let\euscr\mathscr \let\mathscr\relax% just so we can load this and rsfs
\usepackage[scr]{rsfso}
\usepackage{amsthm}
\usepackage{amssymb}
\usepackage{bbm}
\usepackage{multicol}
\usepackage[colorlinks=true, pdfstartview=FitV, linkcolor=blue,
citecolor=blue, urlcolor=blue]{hyperref}

\DeclareMathOperator{\arcsec}{arcsec}
\DeclareMathOperator{\arccot}{arccot}
\DeclareMathOperator{\arccsc}{arccsc}
\newcommand{\ddx}{\frac{d}{dx}}
\newcommand{\dfdx}{\frac{df}{dx}}
\newcommand{\ddxp}[1]{\frac{d}{dx}\left( #1 \right)}
\newcommand{\dydx}{\frac{dy}{dx}}
\let\ds\displaystyle
\newcommand{\intx}[1]{\int #1 \, dx}
\newcommand{\intt}[1]{\int #1 \, dt}
\newcommand{\defint}[3]{\int_{#1}^{#2} #3 \, dx}
\newcommand{\imp}{\Rightarrow}
\newcommand{\un}{\cup}
\newcommand{\inter}{\cap}
\newcommand{\ps}{\mathscr{P}}
\newcommand{\set}[1]{\left\{ #1 \right\}}
\newtheorem*{sol}{Solution}
\newtheorem*{claim}{Claim}
\newtheorem{problem}{Problem}
\pgfplotsset{compat=1.17}
\begin{document}
 
% Don't change the above session

\lhead{Financial Mathematics hw}
\chead{111352027}
\rhead{\today}
\section*{Useful sets}
$\mathcal{F}_0 = \{\phi,\Omega\} \quad\ \mathcal{F}_1 = \{\{\omega_1, \omega_2, \omega_3\},\cdots, \Omega\}\quad\ \mathcal{F}_2 = \{\{\omega_1\},\cdots, \Omega\}\quad\ \mathcal{F}_3\ with\ 64\ items$\\
$A = \{\omega_1, \omega_4\}\quad\ B = \{\omega_1, \omega_2, \omega_3\}\quad\ C = \{\omega_3, \omega_4, \omega_5\}$\\\\
\begin{align*}
    align
\end{align*}
% \maketitle
\section{Section}
\subsection*{subsection}
\begin{enumerate}
    \item example
\end{enumerate}
\subsection*{subsection}
\begin{align*}
   equation
\end{align*}
\section{section}
\subsection*{subsection}
\begin{align*}
    equation
\end{align*}
\subsection*{subsection}
\subsection*{subsection}
\begin{enumerate}
    \item example
\end{enumerate}
\subsection*{subsection}
\begin{align*}
    equation
\end{align*}
\subsection*{subsection}
\begin{align*}
    equation
\end{align*}
\end{document}