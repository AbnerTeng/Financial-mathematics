\documentclass[12pt]{article}
\usepackage{tikz}
\usetikzlibrary{trees}
\usepackage[margin=1in]{geometry} 
\usepackage{amsmath,amsthm,amssymb,scrextend}
\usepackage{fancyhdr}
\setlength{\headheight}{14.5pt}
\addtolength{\topmargin}{-2.5pt}
\pagestyle{fancy}

\newcommand{\cont}{\subseteq}
\usepackage{tikz}
\usepackage{pgfplots}
\usepackage{amsmath}
\usepackage[mathscr]{euscript}
\let\euscr\mathscr \let\mathscr\relax% just so we can load this and rsfs
\usepackage[scr]{rsfso}
\usepackage{amsthm}
\usepackage{amssymb}
\usepackage{bbm}
\usepackage{multicol}
\usepackage[colorlinks=true, pdfstartview=FitV, linkcolor=blue,
citecolor=blue, urlcolor=blue]{hyperref}

\DeclareMathOperator{\arcsec}{arcsec}
\DeclareMathOperator{\arccot}{arccot}
\DeclareMathOperator{\arccsc}{arccsc}
\newcommand{\ddx}{\frac{d}{dx}}
\newcommand{\dfdx}{\frac{df}{dx}}
\newcommand{\ddxp}[1]{\frac{d}{dx}\left( #1 \right)}
\newcommand{\dydx}{\frac{dy}{dx}}
\let\ds\displaystyle
\newcommand{\intx}[1]{\int #1 \, dx}
\newcommand{\intt}[1]{\int #1 \, dt}
\newcommand{\defint}[3]{\int_{#1}^{#2} #3 \, dx}
\newcommand{\imp}{\Rightarrow}
\newcommand{\un}{\cup}
\newcommand{\inter}{\cap}
\newcommand{\ps}{\mathscr{P}}
\newcommand{\set}[1]{\left\{ #1 \right\}}
\newtheorem*{sol}{Solution}
\newtheorem*{claim}{Claim}
\newtheorem{problem}{Problem}
\pgfplotsset{compat=1.17}
\begin{document}
 
% Don't change the above session

\lhead{Financial Mathematics hw2}
\chead{111352027}
\rhead{\today}
\section*{Useful sets}
$\mathcal{F}_0 = \{\phi,\Omega\} \quad\ \mathcal{F}_1 = \{\{\omega_1, \omega_2, \omega_3\},\cdots, \Omega\}\quad\ \mathcal{F}_2 = \{\{\omega_1\},\cdots, \Omega\}\quad\ \mathcal{F}_3\ with\ 64\ items$\\
$A = \{\omega_1, \omega_4\}\quad\ B = \{\omega_1, \omega_2, \omega_3\}\quad\ C = \{\omega_3, \omega_4, \omega_5\}$\\\\
\begin{align*}
    X_1(\omega_i)=\begin{cases}
        1, & i = 1,2,3,4,5,6
    \end{cases}\quad\ 
    X_2(\omega_i)=\begin{cases}
        2, & i = 1,4\\
        4, & i = 2,3,5,6
    \end{cases}\quad\ \\
    X_4(\omega_i)=\begin{cases}
        1, & i = 1,2,3\\
        3, & i = 4,5,6
    \end{cases}\quad\ 
    X_5(\omega_i)=\begin{cases}
        i, & i = 1,2,3,4,5,6
    \end{cases}
\end{align*}
% \maketitle
\section{Find Conditional Probability}
\subsection*{Conidtional Probability}
\begin{enumerate}
    \item $P(A|\mathcal{F}_0)=P(A|\{\phi,\Omega\}) = P(\{\omega_1, \omega_4\}|\{\phi,\Omega\}) = \dfrac{1}{3}$ 
    \item $P(C|\mathcal{F}_1) = P(C|\{\{\omega_1, \omega_2, \omega_3\},\cdots, \Omega\}) = P(\{\omega_3, \omega_4, \omega_5\}|\{\{\omega_1, \omega_2, \omega_3\},\cdots, \Omega\}) = \dfrac{5}{12}$
    \item $P(A|\mathcal{F}_2) = P(A|\{\{\omega_1\},\cdots, \Omega\}) = P(\{\omega_1, \omega_4\}|\{\{\omega_1\},\cdots, \Omega\}) = 1$ 
\end{enumerate}
\subsection*{\begin{proof}
    $A\in \mathcal{F}_2\Rightarrow P(A|\mathcal{F}_2) = \mathbbm{1}_A$
\end{proof}}
\begin{align*}
    P(A|\mathcal{F}_2)(\omega_1) = P(A|\mathcal{F}_2)(\omega_4) = 1 \Rightarrow P(A|\mathcal{F}_2)=1=\mathbbm{1}_A\\
\end{align*}
\section{Find conditional expectation}
\subsection*{\begin{proof}
    $E(X_2|\mathcal{F}_1)$ is an expectation of $X_4$ on $\mathcal{F}_1$
\end{proof}}
\begin{align*}
    E(X_2|\mathcal{F}_1)(\omega_1) &= E(X_2|\mathcal{F}_1)(\omega_4) = 2\\
    E(X_2|\mathcal{F}_1)(\omega_3) &= \cdots = E(X_2|\mathcal{F}_1)(\omega_6) = 4\\
    E(X_2|\mathcal{F}_1) &= 2\times \dfrac{1}{3} + 4\times \dfrac{2}{3} = \dfrac{10}{3}
\end{align*}
\subsection*{Find $E(X_4)$}
According to Law of Iterated Expectation (L.I.E), $E(X_4) = E[E(X_4|\mathcal{F}_1)]$. We know that $X_4$ 
is measurable in $\mathcal{F}_1$, so $E(X_4|\mathcal{F}_1) = X_4 = 2$. So $E(X_4) = E[E(X_4|\mathcal{F}_1)] = E(2) = 2$
\subsection*{Conditional expectation}
\begin{enumerate}
    \item $E(X_5|\mathcal{F}_2)$\\
    \item $E(X_0|\mathcal{F}_2)$\\
    \item $E(X_4|\mathcal{F}_0)$\\
    \item $E(X_4|\mathcal{F}_2)$\\
    \item $E(X_4|\mathcal{F}_3)$\\
    \item $E(X_4|X_2=4) = E(X_4|X_2(\omega_i), i = 2,3,5,6) = \dfrac{1\times5 + 3\times 3}{8} = \dfrac{7}{4}$
\end{enumerate}
\subsection*{Varify $E(X_2X_4|\mathcal{F}_1) = X_4E(X_2|\mathcal{F}_1)$}
\begin{align*}
    E(X_2X_4|\mathcal{F}_1) &= E(X_2|\mathcal{F}_1)E(X_4|\mathcal{F}_1)
\end{align*}
We know that $X_4$ is measurable in $\mathcal{F}_1$, so $E(X_4|\mathcal{F}_1) = X_4$
\begin{align*}
    E(X_2|\mathcal{F}_1)E(X_4|\mathcal{F}_1) = X_4E(X_2|\mathcal{F}_1)
\end{align*}
\subsection*{\begin{proof}
    $E(X|\mathcal{F}_0) = E(X)$
\end{proof}}
If $X$ is $\mathcal{F}_0$ measurable, that $X$ must be $\mathcal{G}$ measurable. According to Law of Iterated Expectation (L.I.E)
\begin{align*}
    E(X|\mathcal{F}_0) = E[E(X|\mathcal{F}_0)|\mathcal{G}] = E[E(X|\mathcal{G})|\mathcal{F}_0]= E[E(X|\mathcal{G})] = E(X)
\end{align*}
\end{document}