\documentclass[12pt]{article}
\usepackage{tikz}
\usetikzlibrary{trees}
\usepackage[margin=1in]{geometry} 
\usepackage{amsmath,amsthm,amssymb,scrextend}
\usepackage{fancyhdr}
\setlength{\headheight}{14.5pt}
\addtolength{\topmargin}{-2.5pt}
\pagestyle{fancy}
\usepackage{graphicx}
\usepackage{float}
\newcommand{\cont}{\subseteq}
\usepackage{tikz}
\usepackage{pgfplots}
\usepackage{amsmath}
\usepackage[mathscr]{euscript}
\let\euscr\mathscr \let\mathscr\relax% just so we can load this and rsfs
\usepackage[scr]{rsfso}
\usepackage{amsthm}
\usepackage{amssymb}
\usepackage{bbm}
\usepackage{multicol}
\usepackage[colorlinks=true, pdfstartview=FitV, linkcolor=blue,
citecolor=blue, urlcolor=blue]{hyperref}

\DeclareMathOperator{\arcsec}{arcsec}
\DeclareMathOperator{\arccot}{arccot}
\DeclareMathOperator{\arccsc}{arccsc}
\newcommand{\ddx}{\frac{d}{dx}}
\newcommand{\dfdx}{\frac{df}{dx}}
\newcommand{\ddxp}[1]{\frac{d}{dx}\left( #1 \right)}
\newcommand{\dydx}{\frac{dy}{dx}}
\let\ds\displaystyle
\newcommand{\intx}[1]{\int #1 \, dx}
\newcommand{\intt}[1]{\int #1 \, dt}
\newcommand{\defint}[3]{\int_{#1}^{#2} #3 \, dx}
\newcommand{\imp}{\Rightarrow}
\newcommand{\un}{\cup}
\newcommand{\inter}{\cap}
\newcommand{\ps}{\mathscr{P}}
\newcommand{\set}[1]{\left\{ #1 \right\}}
\newtheorem*{sol}{Solution}
\newtheorem*{claim}{Claim}
\newtheorem{problem}{Problem}
\pgfplotsset{compat=1.17}
\usepackage{listings}
\definecolor{dkgreen}{rgb}{0,0.6,0}
\definecolor{codegrey}{rgb}{0.5,0.5,0.5}
\definecolor{codepurple}{rgb}{0.58,0,0.82}
\lstset{
frame=tb,
language=Python,
aboveskip=3mm,
belowskip=3mm,
showstringspaces=false,
columns=flexible,
basicstyle={\small\ttfamily},
numbers=none,
numberstyle=\tiny\color{green},
keywordstyle=\color{blue},
commentstyle=\color{dkgreen},
stringstyle=\color{codepurple},
breaklines=true,
breakatwhitespace=true,
tabsize=3
}
\begin{document}
 
% Don't change the above session

\lhead{Financial Mathematics hw}
\chead{111352027}
\rhead{\today}
%%\section*{Useful sets}
%%$\mathcal{F}_0 = \{\phi,\Omega\} \quad\ \mathcal{F}_1 = \{\{\omega_1, \omega_2, \omega_3\},\cdots, \Omega\}\quad\ \mathcal{F}_2 = \{\{\omega_1\},\cdots, \Omega\}\quad\ \mathcal{F}_3\ with\ 64\ items$\\
%%$A = \{\omega_1, \omega_4\}\quad\ B = \{\omega_1, \omega_2, \omega_3\}\quad\ C = \{\omega_3, \omega_4, \omega_5\}$\\\\
%%\begin{align*}
%%    align
%%\end{align*}
% \maketitle
\section{It\^{o} integral (I)}
\begin{align*}
    E\Bigg[\sum_{i=1}^n X_{t_{i-1}} \Delta W_{t_{i-1}}\cdot\sum_{i=1}^n Y_{t_{i-1}} \Delta W_{t_{i-1}}\Bigg] = \sum_{i=1}^n E\Big[X_{t_{i-1}}Y_{t_{i-1}}\Delta W^2_{t_{i-1}}\Big]
\end{align*}
\begin{align*}
    = \sum_{i=1}^n E \Big[X_{t_{i-1}}Y_{t_{i-1}} E[\Delta W^2_{t_{i-1}}|\mathcal{F}_t]\Big] = \sum_{i=1}^n E\Big[X_{t_{i-1}}Y_{t_{i-1}}\Delta t\Big]
\end{align*}
\section{It\^{o} integral (II)}
\begin{align*}
    Var\Bigg(\int_{0}^t X_s dWs\Bigg) = E\Bigg[\Bigg(\int_0^t X_s dW_s\Bigg)^2\Bigg] = E\Bigg[\int_0^t X_S dW_s \cdot \int_0^t X_S dW_s \Bigg]
\end{align*}
\begin{align*}
    = \int_{0}^t E\Big(X_s^2 dW_s^2\Big) = \int_0^t E\Bigg[X_s^2E\Big(dW_s^2|\mathcal{F}_s\Big)\Bigg] = \int_0^t E\Big(X_s^2\Big)dS = X_t^2
\end{align*}
\section{Brownian Motion simulation}
\begin{lstlisting}[language = Python]
import numpy as np
import pandas as pd
from math import sqrt
from pylab import plot, show, grid, xlabel, ylabel, title
## loop data
k = 10000
## year data
T_year = 1
N = 250
h = T_year/N
mu = 0.1
var_year = 0.25
X0 = 0
## daily data
T_day = 1/250
var_day = var_year/sqrt(250)
def BM(N, h, var_year):
    dt = h
    random_increments = np.random.normal(0, 1*var_year, N)*sqrt(dt)
    brownian_motion = np.cumsum(random_increments)
    brownian_motion = np.insert(brownian_motion, 0, 0)
    return brownian_motion, random_increments
def BM_with_drift(mu, N, h):
    W, _ = BM(N, h, var_year)
    dt = h
    time_steps = np.linspace(0, T_year, N+1)
    X = mu*time_steps + W
    return X
for i in range(k):
    X = BM_with_drift(mu, N, h)
    plot(X)
xlabel('t', fontsize = 16)
ylabel('X', fontsize = 16)
title("Brownian Motion with Drift", fontsize = 16)
grid(True)
show()
\end{lstlisting}
\begin{figure}[H]
    \centering
    \includegraphics[width = 0.7\textwidth]{/Users/abnerteng/Desktop/BM_year.png}
\end{figure}
\section*{Brownian Motion simulation with daily parameters}
\begin{lstlisting}
import numpy as np
import pandas as pd
from math import sqrt
from pylab import plot, show, grid, xlabel, ylabel, title
## loop data
k = 10000
## year data
T_year = 1
N = 250
h = T_year/N
mu = 0.1
var_year = 0.25
X0 = 0
## daily data
T_day = 1/250
var_day = var_year/sqrt(250)

def BM_with_drift_daily(mu, N, h):
    W, _ = BM(N, h, var_day)
    dt = h
    time_steps = np.linspace(0, T_day, N+1)
    X = mu*time_steps + W
    return X

for i in range(k):
    X = BM_with_drift_daily(mu/250, N, h)
    plot(X)
xlabel('t', fontsize = 16)
ylabel('X', fontsize = 16)
title("Brownian Motion with Drift", fontsize = 16)
grid(True)
show()
\end{lstlisting}
\begin{figure}[H]
    \centering
    \includegraphics[width = 0.7\textwidth]{/Users/abnerteng/Desktop/BM_daily.png}
\end{figure}
\end{document}